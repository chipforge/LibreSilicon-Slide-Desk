%%%%%%%%%%%%%%%%%%%%%%%%%%%%%%%%%%%%%%%%%
% Beamer Presentation
% LaTeX Template
% Version 1.0 (10/11/12)
%
% This template has been downloaded from:
% http://www.LaTeXTemplates.com
%
% License:
% CC BY-NC-SA 3.0 (http://creativecommons.org/licenses/by-nc-sa/3.0/)
%
%%%%%%%%%%%%%%%%%%%%%%%%%%%%%%%%%%%%%%%%%

%----------------------------------------------------------------------------------------
%	PACKAGES AND THEMES
%----------------------------------------------------------------------------------------

\documentclass{beamer}
\usepackage[utf8]{inputenc}

\mode<presentation> {

% The Beamer class comes with a number of default slide themes
% which change the colors and layouts of slides. Below this is a list
% of all the themes, uncomment each in turn to see what they look like.

%\usetheme{default}
%\usetheme{AnnArbor}
%\usetheme{Antibes}
%\usetheme{Bergen}
%\usetheme{Berkeley}
%\usetheme{Berlin}
%\usetheme{Boadilla}
%\usetheme{CambridgeUS}
%\usetheme{Copenhagen}
%\usetheme{Darmstadt}
%\usetheme{Dresden}
%\usetheme{Frankfurt}
%\usetheme{Goettingen}
%\usetheme{Hannover}
%\usetheme{Ilmenau}
%\usetheme{JuanLesPins}
%\usetheme{Luebeck}
\usetheme{Madrid}
%\usetheme{Malmoe}
%\usetheme{Marburg}
%\usetheme{Montpellier}
%\usetheme{PaloAlto}
%\usetheme{Pittsburgh}
%\usetheme{Rochester}
%\usetheme{Singapore}
%\usetheme{Szeged}
%\usetheme{Warsaw}

% As well as themes, the Beamer class has a number of color themes
% for any slide theme. Uncomment each of these in turn to see how it
% changes the colors of your current slide theme.

%\usecolortheme{albatross}
%\usecolortheme{beaver}
%\usecolortheme{beetle}
%\usecolortheme{crane}
%\usecolortheme{dolphin}
%\usecolortheme{dove}
%\usecolortheme{fly}
%\usecolortheme{lily}
%\usecolortheme{orchid}
%\usecolortheme{rose}
%\usecolortheme{seagull}
%\usecolortheme{seahorse}
%\usecolortheme{whale}
%\usecolortheme{wolverine}

%\setbeamertemplate{footline} % To remove the footer line in all slides uncomment this line
%\setbeamertemplate{footline}[page number] % To replace the footer line in all slides with a simple slide count uncomment this line

%\setbeamertemplate{navigation symbols}{} % To remove the navigation symbols from the bottom of all slides uncomment this line
}

\usepackage{graphicx} % Allows including images
\usepackage{booktabs} % Allows the use of \toprule, \midrule and \bottomrule in tables

%----------------------------------------------------------------------------------------
%	TITLE PAGE
%----------------------------------------------------------------------------------------

\title[Libre Silicon]{LibreSilicon - Bootstrap the Silicon Manufacture Process} % The short title appears at the bottom of every slide, the full title is only on the title page

\author{Hagen SANKOWSKI} % Your name
\institute[Chipforge] % Your institution as it will appear on the bottom of every slide, may be shorthand to save space
{
Chipforge\\ % Your institution for the title page
\medskip
\textit{hsank@nospam.chipforge.org} % Your email address
}
\date{2019-03-15} % {\today} % Date, can be changed to a custom date

\begin{document}

\begin{frame}
\titlepage % Print the title page as the first slide
\end{frame}

%----------------------------------------------------------------------------------------
%	PRESENTATION SLIDES
%----------------------------------------------------------------------------------------

%------------------------------------------------
\begin{frame}
Current Situation
\end{frame}
%------------------------------------------------

\begin{frame}{Current Situation}
Image you like to manufacture your own Chip.
\begin{itemize}
\item You're going to a Foundry,
\item signing at least 3 NDAs (Non-disclosure Agreements), one for the Process Kit, one for the Standard Cell Libary and one for Purchase details,
\item invest a lot of money for the Layout development and the Mask Set,
\item and have some reasons to change the Foundry Service ..
\end{itemize}
\end{frame}

\begin{frame}{Current Situation}
.. you're lost
\end{frame}

\begin{frame}{Current Situation}
Reasons are:
\begin{itemize}
\item the technology is completely different,
\item the Standard Cells are mostly different,
\item the mask does not leave the foundry,
\item nearly nothing match another technology in another foundry.
\item Well, you've burned the costs for layout and mask set.
\end{itemize}
\end{frame}

%------------------------------------------------
\begin{frame}
What to do??
\end{frame}
%------------------------------------------------

\begin{frame}
Make your self independend!
\begin{itemize}
\item design a open and free process.
\item You can help if you like :-)
\end{itemize}
\end{frame}

%------------------------------------------------
\begin{frame}{Libre Silicon}
\begin{itemize}
\item is a free and open, community based Silicon Manufacturing Process, without NDAs.
\item mainly Standard CMOS, plus bipolar, plus flash, plus other useful analog stuff.
\end{itemize}
\end{frame}
%------------------------------------------------

\begin{frame}{Libre Silicon}
What happens so far?
\end{frame}

\begin{frame}{Libre Silicon}
2017:
\begin{itemize}
\item David Lanzendörfer opens a possibility to rent a Clean Room at Hong Kong University of Science and Technology,
\item and got some foundations.
\item At the 34. Chaos Communication Congress in Leipzig he gave a Lightning Talk.
\end{itemize}
\end{frame}

\begin{frame}{Libre Silicon}
2018:
\begin{itemize}
\item We developed the first Version of our 1 micron Libre Silicon process.
\item We are working on the Standard Cell Library.
\item We already hold a Tool Chain Hackathon.
\item We design a first Test Wafer for technology parameter measurement.
\item We present our progress (and 1st Wafer) at the 35. Chaos Communication Congress in Leipzig.
\end{itemize}
\end{frame}

\begin{frame}{Libre Silicon}
2019:
\begin{itemize}
\item Currently we struggle through the Wafer processing, again and again.
\end{itemize}
\end{frame}

\begin{frame}{Libre Silicon}
Links:
\begin{itemize}
\item Process
https://github.com/libresilicon/libresiliconprocess
\item Test Wafer
https://github.com/chipforge/PearlRiver
\item Standard Cell Library
https://github.com/chipforge/StdCellLib
\item Tool Chain
https://github.com/leviathanch/qtflow
\end{itemize}
\end{frame}


%------------------------------------------------
\begin{frame}
What still left?
\end{frame}
%------------------------------------------------

\begin{frame}{Targeting}
\begin{itemize}
\item Documentation about what and how we like to measure Parameters
\item Doing Measurement at HKUST
\item Transfer Parameters into Spice BSIM3v3 models
\item Finishing Standard Cells
\item Manufacture first Chips ("555" and Microcontroller) this year
\item Install our process at foreign Foundries for mass production
\end{itemize}
\end{frame}

\begin{frame}{Targeting}
License:
\begin{itemize}
\item Free and Open Source - while for real Hardware GPL or BSD does not work.
\item Others like CERN we already evaluated.
\item We like that everybody can use the Process (even in your Basement),
\item including Universities and real foundries.
\end{itemize}
\end{frame}

\begin{frame}{Targeting}
Being Transfer-able means:
\begin{itemize}
\item Everybody has the possibility to transfer own designs into other foundries.
\item Setting Standards / Reference for Analog Designs (which heavily depends on process parameters).
\item Feasible for Education purposes also, while LibreSilicon is NDA-free.
\item Foundries can compete in production cost and / or corporate.
\end{itemize}
\end{frame}

\begin{frame}{Contact us}
Jour Fix
\begin{itemize}
\item Every Sunday at 21.00 Hong Kong Time (13.00 UTC)
\item we meet us via Mumble at
\item IP 109.109.202.102, Port 64738, Channel IC
\end{itemize}
\end{frame}

%------------------------------------------------

\begin{frame}{Contact us}
Join our Mailing List
\begin{itemize}
\item https://list.o2s.ch/mailman/listinfo/libre-silicon-devel
\end{itemize}
\end{frame}

%------------------------------------------------

\begin{frame}{Thanks!}
	\begin{center}
		\textbf{Merci!} \\
		\textbf{Thank you very much!} \\
	\end{center}
\end{frame}

%------------------------------------------------

\begin{frame}
\begin{itemize}
\item Mailing List
https://list.o2s.ch/mailman/listinfo/libre-silicon-devel
\item Process
https://github.com/libresilicon/libresiliconprocess
\item Test Wafer
https://github.com/chipforge/PearlRiver
\item Standard Cell Library
https://github.com/chipforge/StdCellLib
\item Layout Software
https://github.com/leviathanch/qtflow
\end{itemize}
\huge{\centerline{You can help :-)}}
\end{frame}

%------------------------------------------------

\begin{frame}
\huge{\centerline{The End}}
\end{frame}

\end{document}
